\documentclass[11pt]{article}
\usepackage[danish]{babel} % danske bogstaver
\usepackage{graphicx} % billeder
\usepackage{float} % H
\usepackage{amsmath} % brøker
\usepackage{listings} % listning af kode
\usepackage{enumerate} % for at anvende egne betegnelser på lister
\usepackage{url} % for at anvende URL i referencer
\usepackage{subcaption} % for at have to billeder/figurer ved siden af hinanden
\usepackage{caption} % for at have to tabeller ved siden af hinanden
\begin{document}

\begin{titlepage}
\title{Skabelon til bachelor projekt}
\author{Frederikke Simone Koefoed Nilsson}
\maketitle
\thispagestyle{empty} % sørger for at sidetal er fjernet fra forside
\end{titlepage}

\tableofcontents
\clearpage % rykker alt indhold efter indholdsfortegnelse til næste side

\section{Grafisk}
I dette afsnit vises hvordan man behandler billeder ift. billedtekst, etiketter og centeres.

\subsection{Billedetekst}
Man kan tilføje billedetekst til et billede ved at anvende \textbf{caption}. Alt efter hvor \textbf{caption} placeres vil billedeteksten enten være under eller over billedet. 

\subsubsection{Billedetekst under billede}
\begin{figure}[H] % H betyder at billede skal sættes HER i teksten
\centering
\includegraphics[width=2cm]{carlaogbella}
\caption{Billedetekst under billede}
\end{figure}

\subsubsection{Billedetekst over billede}
\begin{figure}[H] 
\centering
\caption{Billedetekst over billede}
\includegraphics[width=2cm]{carlaogbella}
\end{figure}

\subsection{Etiket}
Man kan tilføje en etiket/label på et billede i tilfælde af, at man ønsker at henvise til figuren senere hen.

\begin{figure}[H] 
\centering
\includegraphics[width=2cm]{carlaogbella}
\caption{Dette billede indeholder label/etiket}
\label{fig:carlaogbella} % denne figur henvisning til gennem fig:fcarlaogbella
\end{figure}
\subsection{Centrering}
Der er flere måder man kan centrerer et billede på.\\\\
\textbf{Mulighed 1:}
\begin{center}
\includegraphics[width=2cm]{carlaogbella}
\end{center}
\textbf{Mulighed 2:}
\begin{figure}[H]
\centering
\includegraphics[width=2cm]{carlaogbella}
\end{figure}

\subsection{To billeder ved siden af hinanden}
\begin{figure}[H]
\begin{subfigure}{0.5\textwidth}
\centering
\includegraphics[width=2cm]{carlaogbella}
\caption{Carla og Bella 1}
\label{fig:hunde1} 
\end{subfigure}
\begin{subfigure}{0.5\textwidth}
\centering
\includegraphics[width=2cm]{carlaogbella}
\caption{Carla og Bella 2}
\label{fig:hunde2} 
\end{subfigure}
\label{fig:image}
\end{figure}

\section{Henvisning til billede}
Man kan henvise til en figur ved at anvende: backslash + ref. Dog er det kun nummeret på figuren som fremgår og man skal derfor selv husk at skrive "figur" for at det giver mening.\\\\
\textbf{Eksempel:}\\ 
I denne sætning henviser jeg til figur \textbf{\ref{fig:carlaogbella}}, som indeholder billedet af mig.

\section{Henvisning til side, som indeholder billedet}
Man kan henvise til den side, som indeholder det billede/figur ved at anvende: backslash + pageref. \\\\
\textbf{Eksempel:}\\
I denne sætning henviser jeg til den side \textbf{\pageref{fig:carlaogbella}}, som indeholder billedet af mig.
\clearpage
\section{Afsnit}
Det er muligt at anvende diverse afsnit-former. Hvis man anvender et udtryk, som indeholder section vil det højest sandsynligt blive en del af indholdsfortegnelsen. Man anvender \textbf{section}, når man ønsker et nyt afsnit og \textbf{subsection}, når man ønsker et under afsnit. Derudover er det muligt, at anvende \textbf{subsubsection}, hvis man ønsker et yderligere underafsnit. Udover at anvende section, subsection og subsubsection er det muligt, at anvende \textbf{paragraph}, hvis man ønsker at lave en paragraf, samt en \textbf{subparagraph}, hvis man ønsker at lave en under paragraf.

\subsection{Nummereret afsnit}
Hvis man ønsker at sine afsnit er nummererede skal man ikke gøre yderligere.

\subsection*{Ikke-nummereret afsnit}
Hvis man ønsker at sine afsnit ikke er nummerede skal man anvende \textbf{* efter section}. 
\clearpage

\section{Lister}
I dette afsnit vil jeg gennemgå diverse liste-typer, heraf tal, bogstave, punkter m.m.
\subsection{Punkter}
Man kan lave en punkttegns liste ved at anvende \textbf{itemize} og gøre følgende:
\begin{itemize}
\item Frederikke
\item Simone
\item Thomas
\item Søren
\end{itemize}
\subsection{Tal}
Man kan lave en nummereret liste ved at anvende \textbf{enumerate} og gøre følgende: 
\begin{enumerate}
\item Frederikke
\item Simone
\item Thomas
\item Søren
\end{enumerate}
\subsection{Bogstaver}
Man kan selv bestemme hvordan ens liste skal tælle op ad ved at tilføje den ønskede type i parentes - dvs. det er muligt at anvende andet end bogstaver. \\\\
\textbf{Mulighed 1:}
\begin{enumerate}[A.]
\item Frederikke
\item Simone
\item Thomas
\item Søren
\end{enumerate}
\vspace{0.5cm}
\textbf{Mulighed 2:}
\begin{enumerate}[a.]
\item Frederikke
\item Simone
\item Thomas
\item Søren
\end{enumerate}
\subsection{2-i-1}
\begin{enumerate}
\item Mandag
	\begin{itemize}
	\item Skole 08:30-12:00
	\item Mødes med studiegruppe klokken 16
	\end{itemize}
\item Tirsdag
\item Onsdag
\item Torsdag
\item Fredag
	\begin{itemize}
	\item Ud og danse
	\end{itemize}
\item Lørdag
\item Søndag
\end{enumerate}
\clearpage

\section{Tabeller}
I dette afsnit vil jeg vise hvordan man laver diverse tabeller. Det er muligt at finde tabeller som både indeholder sætninger og tal. 

\subsection{Centeret tabel med forskellige vandrette justeringer}
\begin{table}[H]
\centering
\begin{tabular}{|p{0.5cm}|p{2cm}|p{1cm}|p{6cm}|} % p står for paragraph 
\hline
Nr&Navn&Alder&Beskrivelse\\ \hline %hline is a vandret linje (horonzontal)
1&Frederikke&25 år &Læser professionsbachelor i softwareudvikling på Copenhagen Business Academy.\\ \hline
2&Malene&23 år&Læser professionsbachelor i webudvikling på KEA. \\ \hline
\end{tabular}
\end{table}

\subsection{Celle som spænder over flere kolonner}

\begin{table}[H]
\centering
\begin{tabular}{|p{0.5cm}|p{2cm}|p{1cm}|p{6cm}|} % p står for paragraph 
\hline
\multicolumn{4}{|c|}{Oversigt}\\
\hline
Nr&Navn&Alder&Beskrivelse\\ \hline %hline is a vandret linje (horonzontal)
1&Frederikke&25 år &Læser professionsbachelor i softwareudvikling på Copenhagen Business Academy.\\ \hline
2&Malene&23 år&Læser professionsbachelor i webudvikling på KEA. \\ \hline
\end{tabular}
\end{table}

\subsection{Lodret justering i celler}
Hvis der er behov for at justere lodret i en celle kan man anvende \textbf{arraystretch}. Et eksempel på dette kunne være at tabellen indeholdt brøker. \\\\

\begin{table}[H]
\begin{subtable}[h]{0.45\textwidth}
\centering
\begin{tabular}{|c||c|c|c|c|c|} % c stands for centeret r for right aligned l for left aligned
\hline
$x$&1&2&3&4&5\\ \hline %hline is a vandret linje (horonzontal)
$f(x)$&\(\frac{1}{2}\)&11&12&13&14\\ \hline
\end{tabular}
\caption{Før justering}
\end{subtable}
\begin{subtable}[h]{0.45\textwidth}
\def\arraystretch{1.5} 
\begin{tabular}{|c||c|c|c|c|c|} % c stands for centeret r for right aligned l for left aligned

\hline
$x$&1&2&3&4&5\\ \hline %hline is a vandret linje (horonzontal)
$f(x)$&\(\frac{1}{2}\)&11&12&13&14\\ \hline
\end{tabular}
\caption{Efter justering}
\end{subtable}
\end{table}

\subsection{Tabel beskrivelse og etiket}
Det er muligt at tilføje en tabeltekst under og over en tabel og kan gøres ved at anvende \textbf{caption} over eller under \textbf{tabular}. Derudover er det muligt at tilføje en label/etiket på en tabel, hvis man ønsker at kunne henvise senere hen. Dette gøres ved at anvende \textbf{label}.
\begin{table}[H]
\centering
\def\arraystretch{1.5} 
\begin{tabular}{|c||c|c|c|c|c|} % c stands for centeret r for right aligned l for left aligned
\hline
$x$&1&2&3&4&5\\ \hline %hline is a vandret linje (horonzontal)
$f(x)$&10&11&12&13&14\\ \hline
\end{tabular}
\caption{Tabeltekst under tabel}
\label{tab:tal}
\end{table}


\subsection{Henvisning til tabel}
Man henviser til en tabel gennem \textbf{backslash + ref} efterfulgt af ens etiket, som vil angive tallet på den tabel der henvises til. \\\\
I denne sætning henvises til tabel \textbf{\ref{tab:tal}}.
\clearpage

\section{Kode listning}
Man kan liste sin kode op på to forskellige måder, hvoraf den ene mulighed er gennem \textbf{lstlisting} og den anden mulighed er gennem \textbf{verbatim}.\\\\
\textbf{Mulighed 1:}
\begin{lstlisting}[language=Java]
 public class ArrayListExample1{  
 public static void main(String args[]){  
  ArrayList<String> list=new ArrayList<String>();//Creating arraylist    
      list.add("Mango");//Adding object in arraylist    
      list.add("Apple");    
      list.add("Banana");    
      list.add("Grapes"); 
      
      //Printing the arraylist object   
      System.out.println(list);  
 }  
}  
\end{lstlisting}
\vspace{1cm}
\textbf{Mulighed 2: }
\begin{verbatim}
 public class ArrayListExample1{  
 public static void main(String args[]){  
  ArrayList<String> list=new ArrayList<String>();//Creating arraylist    
      list.add("Mango");//Adding object in arraylist    
      list.add("Apple");    
      list.add("Banana");    
      list.add("Grapes"); 
      
      //Printing the arraylist object   
      System.out.println(list);  
 }  
}  
\end{verbatim}
\section{Matematisk}

\subsection{Vis matematisk udtryk i sætning}
Det er muligt at anvende matematiske udtryk i vores sætninger gennem \textbf{dollar-tegnet}, som gør udtrykket italic. Et eksempel på dette kunne være: $(x+3)$ (bemærk italic). 
\subsection{Vis matematisk udtryk på separat linje}
Det er muligt få sine matematiske udtryk på en separat linje ved at anvende 2 dollar tegn. Et eksempel på dette kunne være $$6x - 2 = 2x$$
\subsection{Brøker, summering, produkt, limit}
\subsubsection{Brøker}
Ved anvendelse af brøker i en sætning skrive det således: \(\frac{1}{2}\). Derudover kan man vælge at lave en displaystyle og dermed vil det vises således: \( \displaystyle \frac{1}{2} \). Hvis man ønsker at brøkeren skal stå alene kan man vælge, at gå med 2 dollar tegn omkring: $$\frac{1}{2}$$
\subsubsection{Summerings notation}
Det er muligt at lave en summering (summation) notation således: $$\sum_{n=1}^{\infty} 2^{-n} = 1$$

\subsubsection{Produkt notation}
Det er muligt at lave en produkt notation således: $$\prod_{i=a}^{b} f(i)$$

\subsubsection{Limit notation}
Det er muligt at lave en limit notation således: $$\lim_{lower}$$

\section{Bibliografi med bog, artikel og internet link}
Den nemmeste måde at lave en bibliografi på er at anvende Google Scholor, som har al information skrevet i bibtex format.

\subsection{Bog}
Dette er en reference på bogen \textit{Discrete Matematics with Applications} \cite{koshy2004discrete}
\subsection{Artikel}
Dette er en reference på en internet artikel \cite{choy2003development}

\subsection{Internet link}
Dette er en reference på internet artiklen hos TV2 \cite{tv2}
\clearpage
\bibliographystyle{plain}
\bibliography{Referencer}

\end{document}
